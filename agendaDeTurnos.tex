\documentclass[a4paper,twoside]{article}
\usepackage{z-eves}
\usepackage[utf8]{inputenc}
\usepackage[margin=2cm]{geometry}
\usepackage{enumitem}
\usepackage{ifthen}

\newcommand{\desig}[2]{\item #1 $\approx #2$}
\newenvironment{designations}
  {\begin{leftbar}
    \begin{list}{}{\setlength{\labelsep}{0cm}
                   \setlength{\labelwidth}{0cm}
                   \setlength{\listparindent}{0cm}
                   \setlength{\rightmargin}{\leftmargin}}}
  {\end{list}\end{leftbar}}

  \newcommand{\setlog}{$\{log\}$\xspace}

  \def\titlerunning{\setlog}
  \def\authorrunning{F. Sansone}

  \title{Trabajo Práctico para Ingeniería de Software}
  \author{Franco Sansone}

  \date{2018}

  \begin{document}
  \thispagestyle{empty}
  \maketitle

  \section{Requerimientos}
  El usuario puede asociar un cliente (paciente si estuviese relacionado a la medicina) por nombre y DNI a un día y horario. Puede dar de baja un turno.

  Se puede filtrar por fecha, nombre y DNI.

  Podría ser usado por cualquiera que desarrolle sus actividades mediante turnos (médicos, docentes que den clases particulares, psicólogos, etc).

  \section{Especificación}
  Comenzamos dando algunas designaciones.

  \begin{zed}
  \desig{$n$ es un nombre}{n \in NAME}
  \desig{$d$ es una fecha}{d \in DATE}
  \desig{$k$ es el nombre de una persona cuyo cumplea�os hay registrar}{k \in known}
  \desig{La fecha de cumplea�os de la persona $k$}{birthday~k}
  \end{zed}

  Entonces introducimos los siguientes tipos básicos.

  \begin{zed}
  [NAME,DATE]
  \end{zed}

  Ahora podemos definir el espacio de estados de la especificación de la siguiente forma.

  \begin{schema}{BirthdayBook}
  known: \power NAME \\
  birthday: NAME \pfun DATE
  \end{schema}



\end{document}
