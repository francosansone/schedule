\documentclass[%
  fleqn,colorlinks,linkcolor=blue,citecolor=blue,urlcolor=blue]{eptcs}
\usepackage[latin1]{inputenc}
\usepackage[spanish]{babel}
\usepackage{amsfonts}
\usepackage{amsmath}
\usepackage{amssymb}
\usepackage{amsthm}
\usepackage{z-eves}
\usepackage{framed}
\usepackage[latin1]{inputenc}
\usepackage[spanish]{babel}
\usepackage{xspace}

\newcommand{\desig}[2]{\item #1 $\approx #2$}
\newenvironment{designations}
  {\begin{leftbar}
    \begin{list}{}{\setlength{\labelsep}{0cm}
                   \setlength{\labelwidth}{0cm}
                   \setlength{\listparindent}{0cm}
                   \setlength{\rightmargin}{\leftmargin}}}
  {\end{list}\end{leftbar}}

  \DeclareMathOperator{\Ima}{Im}

  \newcommand{\setlog}{$\{log\}$\xspace}

  \def\titlerunning{\setlog}
  \def\authorrunning{F. Sansone}

  \title{Trabajo Pr\'actico para Ingenier\'i{}a de Software}
  \author{Franco Sansone}

  \date{2019}

  \begin{document}
  \thispagestyle{empty}
  \maketitle

  \section{Requerimientos}
  El usuario puede asociar un cliente (paciente si estuviese relacionado a la medicina) por nombre y DNI a un d\'i{}a y horario. Puede dar de baja un turno.

  Se puede filtrar por fecha, nombre y DNI.

  Podr\'i{}a ser usado por cualquiera que desarrolle sus actividades mediante turnos (m\'edicos, docentes que den clases particulares, psic\'ologos, etc).

  \section{Especificaci\'on}
  Designaciones.

  \begin{designations}
  \desig{$d$ es un DNI}{d \in DNI}
  \desig{$n$ es un nombre}{n \in NAME}
  \desig{$f$ es una fecha y hora}{f \in DATETIME}
  \desig{El nombre de la persona con DNI $k$}{clientes~k}
  \desig{La fecha y hora del turno de la persona con DNI $k$}{turnos~k}
  \end{designations}

  Entonces introducimos los siguientes tipos b\'asicos.

  \begin{zed}
  [DNI,NOMBRE,DATETIME]
  \end{zed}

  Ahora podemos definir el espacio de estados de la especificaci\'on de la siguiente forma.

  \begin{schema}{AgendaDeTurnos}
  clientes: DNI \pfun NOMBRE \\
  turnos: DATETIME \pfun DNI
  \end{schema}

  El estado inicial de la agenda de turnos es el siguiente.

  \begin{schema}{AgendaDeTurnosInit}
  AgendaDeTurnos
  \where
  clientes = \emptyset \\
  turnos = \emptyset
  \end{schema}

  Primero modelo el esquema que describe los predicados que deber\'i{}an ser invariantes de estado.

  \begin{schema}{AgendaDeTurnosInv}
  AgendaDeTurnos
  \where
  \dom clientes = \Ima turnos
  \end{schema}

  Avanzo con la primera operaci\'on: asignar un turno.

  \begin{schema}{AsignarTurnoPrimeraVez}
  \Delta AgendaDeTurnos \\
  nombre?:NAME \\
  dni?:DNI \\
  fecha?:DATETIME \\
  ahora?:DATETIME \\
  \where
  dni? \notin \dom clientes \\
  fecha? \notin \dom turnos \\
  fecha?>ahora? \\
  clientes' = clientes \cup \{dni? \mapsto name?\} \\
  turnos' = turnos \cup \{fecha? \mapsto dni?\} \\
  \end{schema}

  El argumento de entrada ahora? no necesariamente tiene que ser una entrada ingresada por el usuario.
  El pasaje de este argumento a la subrutina deber\'i{}a ser transparente para el usuario.

  \begin{schema}{AsignarTurnoCliente}
  \Delta AgendaDeTurnos \\
  dni?:DNI \\
  fecha?:DATETIME \\
  ahora?:DATETIME
  \where
  dni? \in \dom clientes \\
  fecha? \notin \dom turnos \\
  fecha? > ahora? \\
  turnos' = turnos \cup \{fecha? \mapsto dni?\} \\
  clientes' = clientes \\
  \end{schema}

  \begin{schema}{ErrorFechaPasada}
  \Xi AgendaDeTurnos \\
  fecha?:DATETIME \\
  ahora?:DATETIME \\
  \where
  fecha? <= ahora? \\
  \end{schema}

  \begin{schema}{TurnoYaAsignado}
  \Xi AgendaDeTurnos \\
  fecha?:DATETIME \\
  \where
  fecha? \in turnos
  \end{schema}

  \begin{zed}
  AsignarTurnoOk  == AsignarTurnoPrimeraVez \lor AsignarTurnoCliente \\
  AsignarTurnoError == ErrorFechaPasada \lor TurnoYaAsignado \\
  AsignarTurno == AsignarTurnoOk \lor AsignarTurnoError
  \end{zed}

  Las siguientes operaciones corresponden a la b\'usqueda de turnos por nombre, DNI y fecha.

  \begin{schema}{BusquedaPorDNIOk}
  \Xi AgendaDeTurnos \\
  dni?:DNI \\
  resp!:\power DATETIME \\
  \where
  dni? \in \dom clientes \\
  resp! = \dom(turnos \dres \{dni?\})
  \end{schema}

  \begin{schema}{ClienteNoExisteDNI}
  \Xi AgendaDeTurnos \\
  dni?:DNI \\
  \where
  dni? \notin \dom clientes
  \end{schema}

  \begin{zed}
  BusquedaPorDNI == BusquedaPorDNIOk \lor ClienteNoExisteDNI
  \end{zed}

  \begin{schema}{BusquedaPorNombreOk}
  \Xi AgendaDeTurnos \\
  nombre?:NOMBRE \\
  resp!: DATETIME \pfun DNI \\
  \where
  nombre? \in \ran clientes \\
  resp! = turnos \rres (\dom (clientes \rres \{nombre?\}))
  \end{schema}

  \begin{schema}{ClienteNoExisteNombre}
  \Xi AgendaDeTurnos \\
  nombre?:NOMBRE \\
  \where
  nombre? \notin \Ima clientes
  \end{schema}

  \begin{zed}
  BusquedaPorNombre == BusquedaPorNombreOk \lor ClienteNoExisteNombre
  \end{zed}

  \begin{schema}{BusquedaPorFecha}
  \Xi AgendaDeTurnos \\
  dia?:\power DATETIME \\
  resp!:DATETIME \pfun DNI
  \where
  resp! = dia? \dres turnos
  \end{schema}

\section{Simulaciones}

La primera simulaci\'on es la siguiente:

\begin{verbatim}
agendaDeTurnosInit(S0) &
asignarTurno(S0,'Franco',37449292,201904011900,201903301900,S1) &
asignarTurno(S1,'Oscar',22099399,201904051730,201903301901,S2) &
asignarTurno(S2,'',22099399,201904011900,201903301903,S3) &
busquedaPorDni(S3,37449292,R1,S4) & busquedaPorNombre(S4,'Oscar',R2,S5).
\end{verbatim}
cuyo primer resultado es:
\begin{verbatim}
S0 = {[clientes,{}],[turnos,{}]},
S1 = {[clientes,{[37449292,Franco]}],[turnos,{[201904011900,37449292]}]},
S2 = {[clientes,{[37449292,Franco],[22099399,Oscar]}],
  [turnos,{[201904011900,37449292],[201904051730,22099399]}]},
S3 = {[clientes,{[37449292,Franco],[22099399,Oscar]}],
  [turnos,{[201904011900,37449292],[201904051730,22099399]}]},
S4 = {[clientes,{[37449292,Franco],[22099399,Oscar]}],
  [turnos,{[201904011900,37449292],[201904051730,22099399]}]},
S5 = {[clientes,{[37449292,Franco],[22099399,Oscar]}],
  [turnos,{[201904011900,37449292],[201904051730,22099399]}]},
S6 = {[clientes,{[37449292,Franco],[22099399,Oscar]}],
  [turnos,{[201904011900,37449292],[201904051730,22099399]}]}
R1 = {201904011900},
R2 = {[201904051730,22099399]},
R3 = {[201904011900,37449292]},
\end{verbatim}

Esta es la segunda simulaci\'on:

\begin{verbatim}
S0={[turnos,
  { [201904011900,37449292],
    [201904011920, 33456787],
    [201904011940, 32890789],
    [201904031900,11235678]
  }],
  { [37449292,'Franco'],
    [33456787,'Ramiro'],
    [11235678,'Florencia']
}} &
busquedaPorFecha(S0,{201904011900,201904011920,201904011940},R,S1).
\end{verbatim}

cuya ejecuci\'on devolvi\'o:

\begin{verbatim}
S0 = {[turnos,
  {[201904011900,37449292],
  [201904011920,33456787],
  [201904011940,32890789],
  [201904031900,11235678]}],
  {[37449292,Franco],[33456787,Ramiro],[11235678,Florencia]}},
R = {[201904011900,37449292],[201904011920,33456787],[201904011940,32890789]},
S1 = {[turnos,{[201904011900,37449292],[201904011920,33456787],
[201904011940,32890789],[201904031900,11235678]}],
{[37449292,Franco],[33456787,Ramiro],[11235678,Florencia]}}
\end{verbatim}

\section{Demostraciones con \setlog}

\paragraph{Primera demostraci\'on con \setlog.}

Demuestro que $AsignarTurno$ preserva el invariante $AgendaDeTurnosInv$, o sea el siguiente teorema:
\begin{theorem}{AsignarTurnoPI}
AgendaDeTurnosInv \land AsignarTurno \implies AgendaDeTurnosInv'
\end{theorem}
el cual en \setlog se escribe de la siguiente forma:
\begin{verbatim}
S = {[clientes,C],[turnos,T]} &
S_ = {[clientes,C_],[turnos,T_]} &
ran(T,It) &
dom(C, It) &
asignarTurno(S,B,K,M,H,S_) &
ran(T_,It_) &
ndom(C_,It_).
\end{verbatim}

\paragraph{Segunda demostraci\'on con \setlog.}
Demuestro que $AsignarTurno$ preserva el invariante $turnos \in DATETIME \pfun DNI$, o sea el teorema:
\begin{theorem}{TurnosIsPfun}
turnos \in  DATETIME \pfun DNI \land AsignarTurno \implies turnos' \in DATETIME \pfun DNI
\end{theorem}
el cual en \setlog se escribe de la siguiente forma:
\begin{verbatim}
S = {[turnos,T],[clientes,C]} &
S_ = {[turnos,T_],[clientes,C_]} &
pfun(T) &
asignarTurno(S,F,J,M,K,S_) &
npfun(T_).
\end{verbatim}

\section{Demostraci\'on con Z/EVES}
Para esta prueba sustitu\'i{} la precondici\'on de $dni? \in \dom clientes$
por $(dni? \mapsto nombre?) \in clientes$ en $AsignarTurnoCliente$ (obviamente,
agregando $nombre? \in NOMBRE$ como par\'ametro de entrada).

% read "/home/fsansone/ingenieria/schedule/zeves.tex";

\begin{theorem}{AsignarTurnoPI}
AgendaDeTurnosInv \land AsignarTurno \implies AsignarTurnoPI'
\end{theorem}

\begin{zproof}[AsignarTurnoPI]
invoke AsignarTurno;
split AsignarTurnoOk;
split AsignarTurnoPrimeraVez;
cases;
prove by reduce;
next;
split AsignarTurnoCliente;
cases;
prove by reduce;
next;
prove by reduce;
next;
\end{zproof}

\section{Casos de prueba}

\end{document}
